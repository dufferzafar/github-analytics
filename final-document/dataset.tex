\section{Dataset}

\subsection{Contents}

The GHTorrent \fnurl{dataset}{http://ghtorrent-downloads.ewi.tudelft.nl/mysql/mysql-2017-09-01.tar.gz} we used is a 60 GB compressed tarball containing 20 CSV files. Each file corresponds to a table of a MySQL database, the schema of which is available \href{http://ghtorrent.org/files/schema.pdf}{here}.
Table 1 lists the CSV files in the dataset.

\definecolor{LightCyan}{rgb}{0.88,1,1}

\vspace{25px}
\begin{table}[htb]
    \centering
    \begin{tabular}{@{}llrl@{}}

        \toprule
             & File & Size & Data \\

        \midrule

           \rowcolor{LightCyan}
            1 & users.csv                   & 1.6 GB   & GitHub Users \\
            2 & organization\_members.csv   & 16.1 MB  & Users that are members of an organization \\
           \rowcolor{LightCyan}
            3 & followers.csv               & 588.7 MB & Users that follow another user \\
           \rowcolor{LightCyan}
            4 & watchers.csv                & 2.9 GB   & Users that watch a project \\
           \rowcolor{LightCyan}
            5 & projects.csv                & 11.1 GB  & GitHub Projects \\
            6 & project\_commits.csv        & 89.8 GB  & Commits on projects (including commits on forks) \\
            7 & project\_languages.cs       & 4.3 GB   & Programming languages used in projects \\
           \rowcolor{LightCyan}
            8 & project\_members.csv        & 491.8 MB & Users that are contributors to projects \\
            9 & repo\_labels.csv            & 6.6 GB   & Labels used in a project \\
           \rowcolor{LightCyan}
           10 & commits.csv                 & 68.8 GB  & Commits on projects \\
           11 & commit\_comments.csv        & 731.6 MB & Comments made on commits \\
           12 & commit\_parents.csv         & 13.7 GB  & Parent(s) of commits \\
           \rowcolor{LightCyan}
           13 & issues.csv                  & 3.0 GB   & GitHub Issues made on projects \\
           14 & issue\_comments.csv         & 4.1 GB   & Comments made on issues \\
           15 & issue\_events.csv           & 4.9 GB   & Actions taken on issues (closing etc.) \\
           16 & issue\_labels.csv           & 262.6 MB & Labels assigned to issues \\
           17 & pull\_requests.csv          & 1.2 GB   & GitHub Pull-requests made on projects \\
           18 & pull\_request\_comments.csv & 2.7 GB   & Comments made on pull-requests \\
           19 & pull\_request\_commits.csv  & 2.3 GB   & Commits made on pull-requests \\
           20 & pull\_request\_history.csv  & 3.5 GB   & Actions taken on pull-requests (merging etc.) \\

        \bottomrule
    \end{tabular}

    \caption{GHTorrent Dataset Contents (\code{mysql-2017-09-01.tar.gz})}
    \small{Colored rows indicate files that we performed analysis on.}
\end{table}

\newpage
\subsection{Preprocessing}

% We used standard UNIX command line tools to
We used \fnurl{xsv}{https://github.com/BurntSushi/xsv/}, a command line CSV parsing tool, to remove columns that contained information we were not interested in. This helped reduce the size of the files. \\

From \code{projects.csv}, we removed the description field which was free form text describing the project. \\

% TODO
\large{ TODO: Other changes that we did! }

\vspace{25px}
\begin{table}[htb]
    \centering

    \begin{tabular}{@{}llrr@{}}
    \toprule
        File & Fields Removed & Old Size & New Size \\
    \midrule
        users.csv            & -                &   1.6 GB &    1.6 GB \\
        followers.csv        & -                & 588.7 MB &  588.7 MB \\
        watchers.csv         & -                &   2.9 GB &    2.9 GB \\
        projects.csv         & Description, URL &  11.1 GB &    4.4 GB \\
        project\_members.csv & ext\_ref\_id     & 491.8 MB &  321.0 MB \\
        commits.csv          & SHA of Commits   &  68.8 GB &   37.6 GB \\
        issues.csv           & Pull Request     &   3.0 GB &    2.8 GB \\
    \midrule
        Total Size           &                  &  88.5 GB &   49.5 GB \\
    \bottomrule
    \end{tabular}

    \caption{Size before and after preprocessing of files}
\end{table}

